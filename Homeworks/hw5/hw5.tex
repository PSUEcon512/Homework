%% LyX 2.3.0 created this file.  For more info, see http://www.lyx.org/.
%% Do not edit unless you really know what you are doing.
\documentclass[10pt]{article}
\usepackage{lmodern}
\usepackage{lmodern}
\usepackage[T1]{fontenc}
\usepackage[utf8]{luainputenc}
\usepackage{geometry}
\geometry{verbose,tmargin=1in,bmargin=1in,lmargin=1in,rmargin=1in}
\usepackage{mathtools}
\usepackage{amsmath}

\makeatletter
%%%%%%%%%%%%%%%%%%%%%%%%%%%%%% User specified LaTeX commands.
 


\usepackage{booktabs}


\usepackage{enumitem}
% \setlist{nosep}

% \usepackage{amsfonts}
% \usepackage{amsmath}
\usepackage{comment}


\usepackage{bbm}
\newcommand{\one}{\mathbbm{1}}
\newcommand{\bs}{\boldsymbol}




\makeatother

\begin{document}
\begin{center}
\textbf{Econ 512}
\par\end{center}

\begin{center}
\emph{Fall 201}9\\[1em]
\par\end{center}

\begin{center}
Homework 5 -- Binary Choice MLE with Random Coefficients\\
 Due 12/02/2019 \\[3em] 
\par\end{center}

Consider the following binary discrete choice model for panel data:
\[
Y_{it}=I\left(\beta_{i}X_{it}+\gamma Z_{it}+u_{i}+\epsilon_{it}>0\right),
\]
where $X_{it}$ and $Z_{it}$ are scalar regressors for a group of
$i=1,...,N$ individuals and $t=1,...,T$ time periods. $\ N=100$
and $T=20$. \ The coefficients $\beta_{i}$ and $u_{i}$ are person
specific and are modeled as draws from a bivariate normal distribution:

\[
\begin{bmatrix}\beta_{i}\\
u_{i}
\end{bmatrix}\sim N(\mu,\Sigma),\quad\text{where}\quad\mu=\begin{bmatrix}\beta_{0}\\
u_{0}
\end{bmatrix},\quad\Sigma=\begin{bmatrix}\sigma_{\beta} & \sigma_{\beta u}\\
\sigma_{\beta u} & \sigma_{u}
\end{bmatrix}.
\]

Assume $\epsilon_{it}$ follows standard logistic distribution, i.e.
$F(\epsilon)=\left(1+e^{-\epsilon}\right)^{-1}$, and then a single
contribution to the likelihood function from individual $i$ is

\[
L_{i}(\gamma\mid\beta_{i},u_{i})=\prod_{t=1}^{T}F(\beta_{i}X_{it}+\gamma Z_{it}+u_{i})^{Y_{it}}\left[1-F(\beta_{i}X_{it}+\gamma Z_{it}+u_{i})\right]^{1-Y_{it}}.
\]

To construct the likelihood function for the data set of NT observations
we have to integrate over the joint distribution of $(\beta_{i},u_{i})$.
\ The likelihood function for the data set is: 
\[
L(\gamma,\mu,\Sigma)=\prod_{i=1}^{N}\int_{-\infty}^{\infty}\int_{-\infty}^{\infty}L_{i}(\gamma\mid\beta_{i},u_{i})\phi(\beta_{i},u_{i}\mid\mu,\Sigma)\,\mathrm{d}\beta_{i}\,\mathrm{d}u_{i},
\]
where $\phi(\cdot\mid\mu,\Sigma)$ is the joint density function of
bivariate normal distribution $N(\mu,\Sigma)$. Of course, it will
be numerically more convenient to work with the log-likelihood function.
The data set \texttt{hw5.mat} contains 20 x 100 matrices of the variables
$X,Z$, and $Y$.
\begin{enumerate}
\item Assume $u_{i}=0\,\,\forall\,i$ (ie. take $u_{i}$ out of the model,
so that $u_{0}=\sigma_{u}=\sigma_{u\beta}=0$). Use Gaussian Quadrature
using 20 nodes to calculate the log-likelihood function when $\beta_{0}=0.1$,
$\sigma_{\beta}=1$, and $\gamma=0$. 
\item Now use Monte Carlo Methods using 100 nodes to calculate the log-likelihood
function. 
\item Maximize (or minimize the negative) log-likelihood function with respect
to the parameters using both integration techniques above. Use Matlab's
fmincon without a supplied derivative to max (min) your objective
function. 
\item Now allow $u_{0}\ne0$, so allow the parameters $\sigma_{u}$ and
$\sigma_{\beta u}$ to be non-zero. Maximize the log-likelihood function,
estimating all of the parameters, using Monte Carlo methods. 
\item For each estimation, report the starting value, argmax, and maximized
value of the log-likelihood function. 
\end{enumerate}
\vspace{2em}
 (Hint: the matlab function ``chol'' may come in handy for simulating
from the joint density.)
\end{document}
