\documentclass[10pt]{article} 
\usepackage[T1]{fontenc} 
\usepackage[utf8]{inputenc}
\usepackage{geometry} 
\geometry{verbose,marginparwidth=0.5in,tmargin=1in,bmargin=1in,lmargin=1in,rmargin=1in} 
\usepackage{lmodern}

\usepackage{booktabs}

\usepackage{enumitem}
% \setlist{nosep}

% \usepackage{amsfonts}
% \usepackage{amsmath}
\usepackage{comment}

\usepackage{mathtools}
\usepackage{bbm}
\newcommand{\one}{\mathbbm{1}}



\begin{document}


\begin{center}
\textbf{Econ 512}

\emph{Fall 2018}\\[1em]

Homework 2 -- Discrete Choice Demand Pricing Equilibrium \\
Due 9/26/2018
\\[3em]
\end{center}

There are two single product firms in a market. There exist a unit mass of consumers. Consumer $i$'s'  utility for product A is

$$u_{iA} = v_A - p_A + \varepsilon_{iA},$$
and her utility for product B is

$$u_{iB} = v_B - p_B + \varepsilon_{iB},$$
where $p_A$ and $P_B$ are the prices and $v_A$ and $v_B$ are the qualities of each product.  

Each consumer chooses to consume a single unit of the product that hives her the highest utility, or the outside option with utility $u_{i0} = \varepsilon_{i0}$.

If $\varepsilon\sim$ iid Extreme Value then consumer demand is the following:

\begin{align*}
D_A &= \frac{exp(q_A - p_A)}{1 + exp(q_A - p_A) + exp(q_B - p_B)} \\
D_B &= \frac{exp(q_B - p_B)}{1 + exp(q_A - p_A) + exp(q_B - p_B)} \\
D_0 &= \frac{1}{1 + exp(q_A - p_A) + exp(q_B - p_B)} 
\end{align*} 

Assume the firms have zero marginal costs and compete by simultaneously setting prices. The equilibrium concept is (Bertrand) Nash. Also, note that 

$$\frac{\partial D_A}{\partial p_A} = -D_A(1-D_A)$$

\vspace{2em}

\noindent
\textbf{1.} Consider the following parameterization: $v_A=v_B=2$. What is the demand for each option if $p_A=p_B=1$?\\[2em]

\noindent
\textbf{2.}
Given the above parameterizations for product values, use Broyden's Method to solve for the Nash pricing equilibrium. (Hint: There is a unique equilibrium.) Report the starting value and convergence criteria (if it converges).\\[2em]

\noindent
\textbf{3.}
Now use a Guass-Sidel method (using the secant method for each sub-iteration) to solve for the pricing equilibirum. Which method is faster? Why?\\[2em]

\noindent
\textbf{3.}
Lastly, use the following update rule to solve for equilibrium:

\begin{equation}
	p^{t+1} = \frac{1}{1-D(p^t)}
\end{equation}

Does this converge? Is it faster or slower than the other two methods?\\[2em]


\noindent
\textbf{5.} Solve the pricing equilibrium (using your preferred method) for $v_A=2$ and $v_B=0:.2:3$. On the same graph, plot equilibrium $p_A$ and $p_B$ as a function of the vector of $v_B$. 





\end{document}