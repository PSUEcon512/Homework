%% LyX 2.3.0 created this file.  For more info, see http://www.lyx.org/.
%% Do not edit unless you really know what you are doing.
\documentclass[10pt]{article}
\usepackage{lmodern}
\usepackage{lmodern}
\usepackage[T1]{fontenc}
\usepackage[utf8]{luainputenc}
\usepackage{geometry}
\geometry{verbose,tmargin=1in,bmargin=1in,lmargin=1in,rmargin=1in}
\usepackage{mathtools}
\usepackage{amsmath}

\makeatletter
%%%%%%%%%%%%%%%%%%%%%%%%%%%%%% User specified LaTeX commands.
 


\usepackage{booktabs}


\usepackage{enumitem}
% \setlist{nosep}

% \usepackage{amsfonts}
% \usepackage{amsmath}
\usepackage{comment}


\usepackage{bbm}
\newcommand{\one}{\mathbbm{1}}

\makeatother

\begin{document}
\begin{center}
\textbf{Econ 512}
\par\end{center}

\begin{center}
\emph{Fall 201}9\\[1em]
\par\end{center}

\begin{center}
Homework 2 -- Discrete Choice Demand Pricing Equilibrium \\
 Due 10/07/2019 \\[3em] 
\par\end{center}

There are two single product firms in a market. There exist a unit
mass of consumers. Consumer $i$'s' utility for product A is

\[
u_{iA}=v_{A}-p_{A}+\varepsilon_{iA},
\]
and her utility for product B is

\[
u_{iB}=v_{B}-p_{B}+\varepsilon_{iB},
\]
where $p_{A}$ and $P_{B}$ are the prices and $v_{A}$ and $v_{B}$
are the qualities of each product.

Each consumer chooses to consume a single unit of the product that
hives her the highest utility, or the outside option with utility
$u_{i0}=\varepsilon_{i0}$.

If $\varepsilon\sim$ iid Extreme Value then consumer demand is the
following:

\begin{align*}
D_{A} & =\frac{exp(v_{A}-p_{A})}{1+exp(v_{A}-p_{A})+exp(v_{B}-p_{B})}\\
D_{B} & =\frac{exp(v_{B}-p_{B})}{1+exp(v_{A}-p_{A})+exp(v_{B}-p_{B})}\\
D_{0} & =\frac{1}{1+exp(v_{A}-p_{A})+exp(v_{B}-p_{B})}
\end{align*}

Assume the firms have zero marginal costs and compete by simultaneously
setting prices. The equilibrium concept is (Bertrand) Nash. Also,
note that

\[
\frac{\partial D_{A}}{\partial p_{A}}=-D_{A}(1-D_{A})
\]

\vspace{2em}

\noindent \textbf{1.} Consider the following parameterization: $v_{A}=v_{B}=2$.
What is the demand for each option if $p_{A}=p_{B}=1$?\\[2em]

\noindent \textbf{2.} Given the above parameterizations for product
values, use Broyden's Method to solve for the Nash pricing equilibrium.
(Hint: There is a unique equilibrium.) Report the starting value and
convergence criteria (if it converges).\\[2em]

\noindent \textbf{3.} Now use a Guass-Sidel method (using the secant
method for each sub-iteration) to solve for the pricing equilibirum.
Which method is faster? Why?\\[2em]

\noindent \textbf{3.} Lastly, use the following update rule to solve
for equilibrium:

\begin{equation}
p^{t+1}=\frac{1}{1-D(p^{t})}
\end{equation}

Does this converge? Is it faster or slower than the other two methods?\\[2em]

\noindent \textbf{5.} Solve the pricing equilibrium (using your preferred
method) for $v_{A}=2$ and $v_{B}=0:.2:3$. On the same graph, plot
equilibrium $p_{A}$ and $p_{B}$ as a function of the vector of $v_{B}$.
\end{document}
