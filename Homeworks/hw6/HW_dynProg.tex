\documentclass[11pt]{article}

\usepackage{natbib,rotfloat,epsfig,setspace,amssymb,amsmath, comment}

\setlength{\hoffset}{0.2in}
\setlength{\voffset}{-0.6in}
\setlength{\textwidth}{6.0in}
\setlength{\evensidemargin}{0in}
\setlength{\oddsidemargin}{0in}
\setlength{\textheight}{8.5in}

\begin{document}

\onehalfspace

\title{Economics 512 -- Homework 6}
\author{Due January 21, 2019}
\date{}
\maketitle
This assignment extends Problem 7.4 in in Miranda \& Fackler. 
Consider competitive price-taking firm that maximizes discounted sum of expected future profits from harvesting a non-renewable resource. 
For example, suppose a lumber company is deciding how many trees to harvest in a forest where trees do not grow back. 
The discount factor of our lumber company is $\delta = 0.95$. The firm earns revenue of $p\cdot x$ per period if it has harvested amount $x$ and market price was $p$. To harvest $x$ the firm incurs a convex cost $0.2\cdot x^{1.5}$. The firm is small relative to the market, and has rational expectations that the price of lumber
will follow an AR(1) process: 
\begin{align}
p_t = p_0 + \rho\cdot p_{t-1} + u
\end{align}
Where $p_0 = 0.5$, $\rho = 0.5$, and $u$ is a mean-zero normal disturbance with standard deviation $\sigma_u=0.1$.
The initial stock of lumber may be anything from 0 to 100.
%\footnote{Note that there is non-zero unconditional probability of price being negative, however it is very unlikely, so we will ignore it.} 
\begin{enumerate}
\item Formulate firm's dynamic optimization problem. Specifically, formulate the Bellman equation, identify state and policy variables, their spaces and transition probabilities. Assume initial stock is between 0 and 100.
\item Take a look at {\tt tauchen.m} in the repository (you should know where), use it to generate grid that approximates process for $p_t$ with 21 grid points. 
\item Solve the firm's problem using value function iteration. Plot the value of the firm depending on its initial stock (x-axis) and the current price of lumber, for $p \in{0.9, 1, 1.1}$.
\item Plot next period optimal stock (or harvest amount if you prefer) as a function of today's price for different amount of lumber left in stock.
\item Assume firm starts with stock of 100 and today's price is 1. Plot expected stock over time for 20 periods ahead. Include the 90 percent confidence interval. 
\item Redo the 2-4 for coarse grid of 5 points in Tauchen's representation.
\item Submit your code together with a pdf of your responses in \LaTeX. (Yes, part of this assignment is to get you to embed figures into \LaTeX.)
\end{enumerate}


\end{document}
